\section{Introduction}
\textit{Warbound} is a turn-based strategy game where players move units on a board to defeat their opponents. The game was developed using Python and the Pygame library.

\subsection{Goal/Objective}
\textit{Warbound} is a turn-based strategy game that involves battlefield tactics and army management. The main objective of the game is to defeat the enemy units. Victory is achieved when a player kills the enemy general or annihilates their entire army. This goal requires a combination of strategic planning, tactical movement, and the effective use of available resources and units.

\subsection{How to play}
This section provides a basic introduction to how to play \textit{Warbound}. For a more detailed tutorial, including the keys to use during the game, refer to the tutorial available in the game.

Initially, each player starts with a set of units on the board, which can be moved each turn. The units vary between infantry, cavalry, and archers, each with its own strengths and weaknesses. Players can adopt different strategies to gain an advantage over their opponents:

\begin{itemize}
    \item \textbf{General Selection:} Before starting a match, each player chooses a general to lead their forces, each with unique abilities that can significantly change the course of the battle. For example, choosing "Alexander the Great" may strengthen infantry units, while "Julius Caesar" may improve the movement and attack capabilities of legions. This initial choice is crucial, as the general defines the playstyle and the initial formation of their army.
    \item \textbf{Formation System:} Units can assume various formations, such as "Phalanx", "Shield Wall", and "Spread", which alter their defense and attack attributes. For instance, the "Phalanx" formation increases attack but reduces defense, ideal for heavy offensives.
    \item \textbf{Unit Orientations:} The orientation of a unit on the battlefield determines its effectiveness in attack and defense. Units facing the enemy directly are more effective in combat, while attacks from the sides or rear can deal more devastating blows.
    \item \textbf{Positioning and Movement:} Utilize the terrain to your advantage by positioning units in strategic locations, such as mountains, which increase defense attributes and can protect more vulnerable units.
\end{itemize}
